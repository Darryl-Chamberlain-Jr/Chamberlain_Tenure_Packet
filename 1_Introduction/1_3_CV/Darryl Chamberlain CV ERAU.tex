\documentclass[10pt,a4paper,sans]{moderncv}       

\moderncvstyle{casual}                 
\moderncvcolor{blue}      
\usepackage{changepage}     
\usepackage[bottom]{footmisc}                                         
% adjust the page margins
\usepackage[left=0.75 in,right= 0.75 in,top= 0.5 in,bottom= 1 in]{geometry}
\setlength{\hintscolumnwidth}{2.4cm}                % if you want to change the width of the column with the dates 

% Counter for Under Review Journal Articles
\newcounter{underReviewArticleCounter}
\newcommand{\underReviewArticle}[1]{%
	\stepcounter{underReviewArticleCounter}
	\cvitem{[\arabic{underReviewArticleCounter}]}{%
		#1
	}
}

% Counter for Peer-Reviewed Journal Articles
\newcounter{peerReviewedArticleCounter}
\newcommand{\peerReviewedArticle}[1]{%
	\stepcounter{peerReviewedArticleCounter}
	\cvitem{[\arabic{peerReviewedArticleCounter}]}{%
		#1
	}
}

% Counter for Peer-Reviewed Conference Articles
\newcounter{peerReviewedConferenceCounter}
\newcommand{\peerReviewedConference}[1]{%
	\stepcounter{peerReviewedConferenceCounter}
	\cvitem{[\arabic{peerReviewedConferenceCounter}]}{%
		#1
	}
}

% Counter for Scholarship in Preparation
\newcounter{inPrepCounter}
\newcommand{\inPrep}[1]{%
	\stepcounter{inPrepCounter}
	\cvitem{[\arabic{inPrepCounter}]}{%
		#1
	}
}

% Counter for Presentations
\newcounter{presentationCounter}
\newcommand{\presentations}[1]{%
	\stepcounter{presentationCounter}
	\cvitem{[\arabic{presentationCounter}]}{%
		#1
	}
}
% Counter for Workshops
\newcounter{workshopCounter}
\newcommand{\workshops}[1]{%
	\stepcounter{workshopCounter}
	\cvitem{[\arabic{workshopCounter}]}{%
		#1
	}
}

% Counter for Invited Talks
\newcounter{invitedTalksCounter}
\newcommand{\invitedTalks}[1]{%
	\stepcounter{invitedTalksCounter}
	\cvitem{[\arabic{invitedTalksCounter}]}{%
		#1
	}
}
% Counter for Session Organizations
\newcounter{sessionOrganizationCounter}
\newcommand{\sessionOrganization}[1]{%
	\stepcounter{sessionOrganizationCounter}
	\cvitem{[\arabic{sessionOrganizationCounter}]}{%
		#1
	}
}

% personal data
\name{Darryl}{Chamberlain Jr.}
\title{Curriculum Vitae}                              
%\address{600 S. Clyde Morris Blvd. Daytona Beach, FL 32114-3900}{}{}          
\email{Darryl.Chamberlain@erau.edu}    
\homepage{faculty.erau.edu/Darryl.Chamberlain}             

\begin{document}
\makecvtitle
\vspace*{-2cm}
\section{Education}
\cventry{2023}
{Certificate, Applied Data Science with Python}
{}{}
{\textit{University of Michigan (Coursera)}}
{ Five-course specialization sequence in Applied Data Science. 
	\href{https://www.coursera.org/account/accomplishments/specialization/certificate/7TEVNZ89JDDA}{\underline{Credential URL}}  
}
{}
\cventry{2012--2017}
	{Ph.D., Mathematics and Statistics}
	{}{}
	{\textit{Georgia State University}}
	{ Research specialization in Collegiate Mathematics Education. \\ 
	\href{https://scholarworks.gsu.edu/math_diss/46/}{\underline{Dissertation}} investigated how students develop an understanding of proof by contradiction. 
	%\\ Advised by Dr. Draga Vidakovic.
	}
	{}
	
\cventry{2007--2010}
	{B.S., Mathematics}
	{University of Florida}
	{}{}{}
\section{Professional Experience}
\cventry{2021--present}
	{Assistant Professor}
	{}
	{Department of Mathematics, Science, \& Technology}
	{Embry-Riddle Aeronautical University -- Worldwide}
	{}
\cventry{2017--2021}
	{Assistant Instructional Professor}
	{}
	{Department of Mathematics}
	{University of Florida}
	{}
\cventry{2013--2017}
	{Graduate Teaching Assistant}
	{}
	{Department of Mathematics and Statistics}
	{Georgia State University}
	{}
\cventry{2011--2012}
	{Teacher}
	{}
	{Mathematics}
	{William T. Dwyer High School, Palm Beach County, FL}
	{} 
\section{Administrative Experience}
\cventry{2023--present}
	{Associate Chair}
	{}
	{Department of Mathematics, Science, \& Technology}
	{Embry-Riddle Aeronautical University -- Worldwide}
	{}
\cventry{2015--2016}
	{Emporium Lab Coordinator}
	{}
	{Department of Mathematics and Statistics}
	{Georgia State University}
	{}
\section{External Research Funding Experience}
\cventry{
	\$382,578\\ funded}
	{\textbf{Co-Principal Investigator}}
	{NSF IUSE: Undergraduate Research for Fully Online STEM Students: Impact of Expanded Curricular Options on STEM Attitudes, Identity, \& Career Ambitions}
	{with Robert Deters (PI), Emily Faulconer (co-PI), Brent Terwilliger (co-PI). 2023-2026.}
	{}{}
\cventry{
	\$233,298\\ funded}
	{\textbf{Co-Principal Investigator}}
	{NSF IUSE: Community of Inquiry and Cognitive Load in Online STEM: Persistence, Performance, and Perspectives}
	{with Emily Faulconer (PI) and Beverly Wood (co-PI). 2021-2024.}
	{}{}
\cventry{
	\$271,543\\ unfunded}
	{\textbf{Principal Investigator}}
	{NSF IUSE: Drilling Down into Concepts with Automatic and Diagnostic Item Generation (Auto-DIG)}
	{with Annie Burns-Childers (co-PI), Catherine Paolucci (co-PI), and Russell Jeter (consult). Submitted October 2020.}
	{}{}
\cventry{
	\$202,184\\ unfunded}
	{\textbf{Co-Principal Investigator}}
	{NSF: Using Video to Expand Communication of Mathematical Sciences Research}
	{with Catherine Paolucci (PI). Submitted October 2020.}
	{}{}
\cventry{
	\$99,960\\ unfunded}
	{\textbf{Principal Investigator}}
	{NSF ECR Core Research: Analyzing a Novel College Algebra Curriculum and Implementation}
	{with Russell Jeter (consult). Submitted October 2019.}
	{}{}
\cventry{
	\$340,764 \\ funded}
	{Graduate Research Assistant (2016--2017); Other Professional (2017--present)}
	{NSF IUSE: Promoting Reasoning in Undergraduate Mathematics (PRIUM)}
	{with Draga Vidakovic (PI), Valerie Miller (Co-PI), and Guantao Chen (Co-PI). 2016-2022.}
	{}{}
\section{Internal Research Funding Experience}
\cventry{
	\$6,000\\ funded}
	{\textbf{Principal Investigator}}
	{ERAU-W Faculty Seed Grant: Collective Knowledge Progression and Proliferation in Asynchronous Calculus Discussion Boards}
	{with Zackery Reed (co-PI) and Karen Keene (co-PI). 2023.}
	{}{}
\cventry{
	\$4,069\\ funded}
	{\textbf{Principal Investigator}}
	{ERAU-W Faculty Seed Grant: Developing Autonomous, Targeted Feedback in Precalculus}
	{2021-2022.}
	{}{}
\cventry{
	\$29,923\\ funded}
	{\textbf{Co-Principal Investigator}}
	{UF Internal Grant: Examining and addressing the content knowledge development needs of Florida's aspiring and newly-qualified mathematics teachers}
	{with Catherine Paolucci (PI) and Christopher Redding (Co-PI). 2020-2021.}
	{}{}
\pagebreak
\section{Journal Articles Under Review}
%\underReviewArticle{%
%	\textbf{Chamberlain Jr., D.} \& Jeter, R. 
%	(under review Dec 2022).
%	\emph{Utilizing theoretically driven distractors to make diagnostic multiple-choice assessments possible}. 
%	International Journal on Research in Undergraduate Mathematics Education Special Issue: Digital Experiences in University Mathematics Education. Advanced and Expectations. 
%}

\underReviewArticle{%
	Reed, Z. \& \textbf{Chamberlain Jr., D.}
	(under review Mar 2023, accepted for chapter submission Jul 2023).
	\emph{A Framework for Analyzing Asynchronous Discussion Activities}. 
	Teaching and Learning Mathematics Online 2e, CRC Press, FL.
}

\underReviewArticle{%
	Paolucci, C., \textbf{Chamberlain Jr., D.}, Redding, C., Vancini, S., \& Reese, A.
	(first submission Nov 2021, revised and resubmitted Aug 2022).
	\emph{Critical lessons from certification exam preparation materials for mathematics teachers' content knowledge and professional learning}. 
	Journal of Teacher Education.
}

\section{Peer-Reviewed Journal Articles}
\peerReviewedArticle{%
	\textbf{Chamberlain Jr., D.} 
	(2023).
	\emph{How one instructor can teach a large-scale, mastery-based College Algebra course online.} 
	Problems, Resources, and Issues in Mathematics Undergraduate Studies. DOI: 10.1080/10511970.2023.2190183.
}
\peerReviewedArticle{%
	Faulconer, E., \textbf{Chamberlain Jr., D.}, \& Woods, B. 
	(2022). 
	\textit{A Case Study of Community of Inquiry Presences and Cognitive Load in Asynchronous Online STEM Courses}.
	Online Learning Journal. DOI: http://dx.doi.org/10.24059/olj.v26i3.3386.
}
\peerReviewedArticle{%
	\textbf{Chamberlain Jr., D.} \& Vidakovic, D. 
	(2021).
	\emph{Cognitive trajectory of proof by contradiction for Transition-to-Proof students}.
	Journal of Mathematical Behavior.
	DOI: 10.1016/j.jmathb.2021.100849.
}
\peerReviewedArticle{%
	\textbf{Chamberlain Jr., D.} \& Jeter, R.\footnotemark \hspace*{1mm} 
	(2020). 
	\emph{Creating diagnostic assessments: Automated distractor generation with integrity}. 
	Journal of Assessment in Higher Education. 
	DOI: 10.32473/jahe.v1i1.116892.
}
\footnotetext{Co-first authors.}
\peerReviewedArticle{%
	\textbf{Chamberlain Jr., D.}, Grady, A., Keeran, S., Knudson, K., Manly, I., Shabazz, M., Stone, C., \& York, A. 
	(2020). 
	\textit{Transitioning to an active learning environment for calculus at the University of Florida.} 
	Problems, Resources, and Issues in Mathematics Undergraduate Studies.
	DOI: 10.1080/10511970.2020.1769235
}
\peerReviewedArticle{
	Stalvey, H., Burns, A., \textbf{Chamberlain Jr., D.}, Kemp, A., Meadows, L., \& Vidakovic, D. 
	(2019). 
	\emph{Students' understanding of the concepts involved in hypothesis testing for one population.} 
	Journal of Mathematical Behavior. 
	DOI: 10.1016/j.jmathb.2018.03.011
}

\section{Peer-Reviewed Conference Proceedings {\footnotesize [asterisk denotes presenter]}}
\peerReviewedConference{
	\textbf{Chamberlain Jr., D.}*, Reed, Z.*, \& Keene, K.
	(2023, Feb 23-25). 
	\textit{Adapting the Argumentative Knowledge Construction Framework to Asynchronous Mathematical Discussions}.
	25th Annual Conference on Research in Undergraduate Mathematics Education: SIGMAA on RUME.
}
\peerReviewedConference{%
	Bailey, T.*, \textbf{Chamberlain Jr., D.}*, \& Christodoulopoulou, K.
	(2022, Feb 24-26). 
	\textit{Undergraduate's covariational reasoning across function representations}.
	24th Annual Conference on Research in Undergraduate Mathematics Education: SIGMAA on RUME.
}
\peerReviewedConference{%
	Reed, Z.*, \textbf{Chamberlain Jr., D.*}, \& Keene, K.
	(2022, Feb 24-26).
	\textit{Argumentative knowledge construction in asynchronous calculus discussion boards}.
	Poster at 24th Annual Conference on Research in Undergraduate Mathematics Education: SIGMAA on RUME, Boston, MA.
}
\peerReviewedConference{%
	Kemp, A.*, \textbf{Chamberlain Jr., D.}, Cooley, L., Miller, V., \& Vidakovic, D. 
	(2020, Feb 27-29). 
	\textit{Student self- and simulated peer-evaluation of proof comprehension: Tina}. 
	23rd Annual Conference on Research in Undergraduate Mathematics Education: SIGMAA on RUME.
}
\peerReviewedConference{%
	\textbf{Chamberlain Jr., D.*} \& Jeter, R. 
	(2019, Feb 28 - Mar 2). 
	\textit{Leveraging cognitive theory to create large-scale learning tools}.
	22nd Annual Conference on Research in Undergraduate Mathematics Education: SIGMAA on RUME.
}
\peerReviewedConference{%
	\textbf{Chamberlain Jr., D.*} \& Vidakovic, D. 
	(2018, Feb 22-24). 
	\emph{Developing proof comprehension and proof by contradiction through logical outlines}. 
	21st Annual Conference on Research in Undergraduate Mathematics Education: SIGMAA on RUME.
}
\peerReviewedConference{%
	Burns, A.*, \textbf{Chamberlain Jr., D.}, Kemp, A.*, Meadows, L., Stalvey, H., \& Vidakovic, D. 
	(2018, Feb 22-24). 
	\emph{Reasoning about one population hypothesis testing: The case of Steve}. 
	21st Annual Conference on Research in Undergraduate Mathematics Education: SIGMAA on RUME.
}
\peerReviewedConference{%
	\textbf{Chamberlain Jr., D.*} \& Vidakovic, D. 
	(2017, Feb 23-25). 
	\emph{Developing student understanding: The case of proof by contradiction}. 
	20th Annual Conference on Research in Undergraduate Mathematics Education: SIGMAA on RUME.
}
\peerReviewedConference{%
	Burns, A.*, \textbf{Chamberlain Jr., D.}, Kemp, A.*, Meadows, L., Stalvey, H., \& Vidakovic, D. 
	(2017, Feb 23-25). 
	\emph{Students' understanding of test statistics in hypothesis testing}. 
	20th Annual Conference on Research in Undergraduate Mathematics Education: SIGMAA on RUME.
}
\peerReviewedConference{%
	Abel, T.*, Brazas, J.*, \textbf{Chamberlain Jr., D.}, \& Kemp, A. 
	(2017, Feb 23-25). 
	\emph{Characterizing mathematical digital literacy: A preliminary investigation}. 
	20th Annual Conference on Research in Undergraduate Mathematics Education: SIGMAA on RUME.
}	
\peerReviewedConference{%
	\textbf{Chamberlain Jr., D.*} \& Vidakovic, D. 
	(2016, Feb 25). 
	\emph{Use of strategic knowledge in a transition-to-proof course: Differences between an undergraduate and graduate student}. 
	19th Annual Conference on Research in Undergraduate Mathematics Education: SIGMAA on RUME. %Retrived from \href{http://sigmaa.maa.org/rume/crume2016/Papers/RUME\_19\_paper\_7.pdf}{http://sigmaa.maa.org/rume/crume2016/Papers/RUME\_19\_paper\_7.pdf}
}
	


\section{Conference Presentations {\footnotesize [asterisk denotes presenter]}}
\presentations{
	\textbf{Chamberlain Jr., D.}*
	(2023, Aug 2). 
	\textit{Technology Use in Undergraduate Mathematics Classrooms}.
	2023 MAA MathFest, Tampa, FL. 
}
\presentations{
	\textbf{Chamberlain Jr., D.}*, Reed, Z.*, Rister, A.*, \& Velez, M.*
	(2023, Feb 7). 
	Roundtable discussion: \textit{Practical Suggestions to Improve Online Discussions Across Disciplines}.
	2023 Academic Innovation Virtual Conference hosted by ERAU-W (virtual). 
}
\presentations{
	Faulconer, E.*, \textbf{Chamberlain Jr., D.*}, \& Woods, B.
	(2022, April 13). 
	\textit{Instructional Efficiency in Asynchronous Online Discussions}.
	Online Learning Consortium Innovate Conference, Dallas, TX.
}
\presentations{%
	Paolucci, C.*, \textbf{Chamberlain Jr., D.}, \& Vancini, S.*
	(2022, Apr 7).
	\textit{Investigating alternatively-certified teachers' mathematical knowledge for teaching calculus}.
	Joint Mathematics Meeting, Seattle, WA.
}
\presentations{%
	\textbf{Chamberlain Jr., D.*}, Reed, Z., \& Keene, K.
	(2021, Nov 20).
	\textit{Investigating social construction of knowledge during asynchronous discussions}.
	5th Northeastern Conference on Research in Undergraduate Mathematics Education. New Brunswick, NJ (virtual).
}
\presentations{%
	Babiceanu, L.* \& \textbf{Chamberlain Jr., D.}
	(2021, Feb 20).
	\textit{Analyzing student achievement with residential and online students in College Algebra}.
	Florida Section of the Mathematical Association of America and Florida Two-Year College Mathematics Association 2021 Joint Meeting, Gainesville, FL (virtual).
}
\presentations{%
	\textbf{Chamberlain Jr., D.*} \& Jeter, R.
	(2021, Jan 7).
	\textit{Automated AF: Leveraging augmented intelligence to provide automated, actionable feedback.}
	Joint Mathematics Meeting, Washington, D.C. (virtual).
}
\presentations{%
	\textbf{Chamberlain Jr., D.*}  \& Jeter, R.
	(2020, Oct 20). 
	\textit{Incorporating Augmented Intelligence to Enhance Learning: Automatic and Diagnostic Item Generation (Auto-DIG)}. 
	STEMpowered Faculty Symposium, Gainesville, FL (virtual).
} 
\presentations{%
	\textbf{Chamberlain Jr., D.*} \& Vidakovic, D.
	(2020, Oct 3).
	\textit{Potential cognitive obstacles to understanding proof by contradiction.}
	4th Northeastern Conference on Research in Undergraduate Mathematics Education. Philadelphia, PA (virtual).
}
\presentations{%
	\textbf{Chamberlain Jr., D.*} 
	(2020, Jul 30). 
	\textit{Drilling down into content with Auto-DIG: Automatic Diagnostic Item Generation.} 
	MAA MathFest, Philadelphia, PA.
	\textit{Session canceled due to COVID-19 pandemic.}
}
\presentations{%
	\textbf{Chamberlain Jr., D.*} 
	(2020, Jan 18). 
	\textit{Mastery-based assessment in a large-enrollment online College Algebra course.}
	Joint Mathematics Meeting, Denver, CO.
}
\presentations{%
	\textbf{Chamberlain Jr., D.}, Knudson, K., Grady, A.*, Keeran, S., Manly, I., Shabazz, M., Stone, C., \& York, A. 
	(2020, Jan 18). 
	\textit{Active Calculus at the University of Florida.} 
	Joint Mathematics Meeting, Denver, CO.
}
\presentations{%
	\textbf{Chamberlain Jr., D.*} \& Jeter, R. 
	(2019, Apr 5). 
	\textit{Creating diagnostic assessments: Automated distractor generation with integrity}. 
	2019 Assessment in Higher Education: Enhancing Institutional Excellence, Gainesville, FL.
}
\presentations{%
	Jeter, R.* \& \textbf{Chamberlain Jr., D.} 
	(2018, Mar 24). 
	\emph{A novel method for creating assessment and diagnostic tools in the classroom}.
	MAA Southeastern Spring Sectional Meeting, Clemson, SC.
}	
\presentations{%
	\textbf{Chamberlain Jr., D.*} \& Vidakovic, D. 
	(2017, Mar 11). 
	\emph{Active learning in transition-to-proof courses: An example lesson of proof by contradiction}. 
	AMS Southeastern Spring Sectional Meeting, Charleston, SC.
}
\presentations{%
	\textbf{Chamberlain Jr., D.*} \& Vidakovic, D. 
	(2017, Jan 5). 
	\emph{A first lesson on proof by contradiction: Developing proof comprehension in a transition-to-proof course}. 
	Joint Mathematics Meeting, Atlanta, GA.
}	
\presentations{%
	\textbf{Chamberlain Jr., D.*}, Kemp, A.*, Meadows, L.*, Stalvey, H., Vidakovic, D., \& Burns, A. 
	(2016, Mar 5). 
	\emph{The emporium model for elementary statistics: A preliminary report}. 
	AMS Southeastern Spring Sectional Meeting, Athens, GA.
}	
\presentations{%
	\textbf{Chamberlain Jr., D.*} \& Vidakovic, D. 
	(2015, Apr 17). 
	\emph{APOS Theory in the classroom}. 
	Center for Instructional Effectiveness Annual Conference, Atlanta, GA.
}	
\presentations{%
	\textbf{Chamberlain Jr., D.*}, Vidakovic, D., Stalvey, H., Burns, A., Meadows, L., \& Kemp, A.* 
	(2015, Apr 10). 
	\emph{Student understanding of one population hypothesis testing: A piece of the process}. 
	Mathematics Graduate Student Miniconference, Atlanta, GA.
}	
\presentations{%
	\textbf{Chamberlain Jr., D.}* \& Vidakovic, D. 
	(2015, Apr 10). 
	\emph{Teaching proofs with APOS Theory}. 
	Mathematics Graduate Student Miniconference, Atlanta, Ga.
}
%\section{Workshops}
%\workshops{}
\section{Invited Talks}
\invitedTalks{%
	\textbf{Chamberlain Jr., D.}
	(2023, Mar 29).
	\textit{Predicting Students' Thoughts to Provide Elaborative Feedback}.
	Invited by California State University Bakersfield Mathematics Department Seminar Series.
}
\invitedTalks{%
	 Faulconer, E., Bourdeau, D., Kiernan, K., \& \textbf{Chamberlain Jr., D.}
	(2023, Jan 21).
	\textit{Non-Traditional Scholarly Publication}. 
	Invited by Embry-Riddle Aeronautical University -- Worldwide Research Scholars Program.
}
\invitedTalks{%
	\textbf{Chamberlain Jr., D.} \& Faulconer, E.
	(2022, Apr 21).
	\textit{How We Manage Large-Scale Data Collection}. 
	Invited by Embry-Riddle Aeronautical University -- Worldwide College of Arts and Sciences Brown Bag Lunch \& Learn Series.
}
\invitedTalks{%
	Paolucci, C. \& \textbf{Chamberlain Jr., D.}
	(2021, Mar 25).
	\textit{A profile of the content knowledge development needs of Florida's alternatively-certified teachers}. 
	Invited by University of Florida Education Policy Research Center Research Brown Bag Series.
}
\invitedTalks{%
	\textbf{Chamberlain Jr., D.} 
	(2020, Nov 13).
	\textit{Integrating Augmented Intelligence into Mathematics Education}. 
	Invited by Florida International University Mathematics Education Seminar.
}
\invitedTalks{%
	\textbf{Chamberlain Jr., D.} 
	(2020, Sept 17).
	\textit{Automatic and Diagnostic Item Generation}. 
	Invited by the University of Florida Lastinger Center.
}

\section{Conference Session/Workshop Organization}
\sessionOrganization{
	\textbf{Chamberlain Jr., D.} \& Barber, R. (2023, Aug 2). Session: \textit{Unspoken Research Components}. 
	2023 MAA MathFest, 
	Tampa, FL.
}
\sessionOrganization{
	\textbf{Chamberlain Jr., D.} \& Barber, R. (2023, Aug 2). Session: \textit{Building a Research Program}. 
	2023 MAA MathFest, 
	Tampa, FL.
}
\sessionOrganization{
	\textbf{Chamberlain Jr., D.}, Reed, Z., \& Keene, K. (2023, Feb 23). Workshop: \textit{Research on Technology in Undergraduate Mathematics Education}. 25th Annual Conference on Research in Undergraduate Mathematics Education: SIGMAA on RUME,
	Omaha, NE. 
}
\sessionOrganization{
	\textbf{Chamberlain Jr., D.}, Acu, B., \& Gasiorek, S. (2023, Jan 3). Session: \textit{Navigating the Early Years of the Faculty Experience}. 
	2023 Joint Mathematics Meeting, 
	Boston, MA.
}
\sessionOrganization{%
	Vidakovic, D., Stalvey, H., \textbf{Chamberlain Jr., D.}, Kemp, A., Meadows, L., \& Kellam, A. (2018, Mar 23-24).
	Session: \textit{Active Learning in Undergraduate Mathematics}. 
	MAA Spring 2018 Southeastern Section Conference, 
	Clemson, SC.
}
\sessionOrganization{%
	Vidakovic, D., Stalvey, H., \textbf{Chamberlain Jr., D.}, Kemp, A., \& Meadows, L. (2017, Mar 10-12).
	Session: \textit{Active Learning in Undergraduate Mathematics}. 
	AMS Spring 2017 Southeastern Regional Conference, 
	Charleston, SC.
}
\sessionOrganization{%
	Vidakovic, D., Stalvey, H., \textbf{Chamberlain Jr., D.}, Kemp, A., \& Meadows, L. (2016, Mar 5-6). 
	Session: \textit{Active Learning in Undergraduate Mathematics}. 
	AMS Spring 2016 Southeastern Regional Conference, 
	Athens, GA.
}

\section{Teaching Experience} 
\cventry{2023--present}
{Introduction to Programming for Data Science}
{Developer/Instructor}{}{}
{\begin{itemize}
		\item Asynchronous online with $10-20$ students.
\end{itemize} }
\cventry{2021--present}
	{Precalculus for Aviation}
	{Instructor}{}{}
	{\begin{itemize}
			\item Asynchronous online with $20-30$ students.
			\item October 2022: EagleVision with 20 students.
		\end{itemize} }
\cventry{2021--present}
	{Precalculus Essentials}
	{Instructor}{}{}
	{\begin{itemize}
			\item Asynchronous online with $20-30$ students.
	\end{itemize} }
\cventry{2018--2021}{Analytic Geometry and Calculus I}{Instructor}{}{}{\begin{itemize}
		\item Fall 2019, Fall 2020: Special flipped class for $\sim$15 Pre-Health PostBac students. 
		\item Summer 2018: Special flipped classroom with $\sim$20 freshmen engineering students.
		\item Spring 2018: Large lecture with 200+ students.
\end{itemize}}
\cventry{Spring 2021}{Sets and Logic}{Instructor}{}{}{\begin{itemize}
		\item Modified Moore's Method with $\sim$30 students. \end{itemize}}
\cventry{Summer 2019}{Analytic Geometry and Calculus II}{Instructor}{}{}{\begin{itemize}
		\item Flipped class with $\sim$20 students. \end{itemize}}
\cventry{Spring 2019}{Elementary Differential Equations}{Instructor}{}{}{\begin{itemize}
		\item Large lecture with 120+ students.
 \end{itemize}}
\cventry{2017--2021}
{College Algebra}
{Developer/Coordinator/Instructor}{}{}
{\begin{itemize}
		\item Multiple sections of Pure Online ($\sim$150 students) and Hyrbid ($\sim$200 students) per semester. 
		\item Curriculum overhaul with focus on understanding of functions.
		\item Developed open-source online homework system/textbook with dynamically-generated problems.
		\item Developed automatically-generated assessments based on students' varying levels of understanding functions.
\end{itemize}}
\cventry{2013--2017}{Various courses}{Instructor of Record as graduate student}{}{}{\begin{itemize}
		\item Elementary Statistics (flipped, $\sim$40 student sections).
		\item Intermediate Algebra (traditional, $\sim$20 student section).
		\item College Algebra (flipped, $\sim$40 student sections).
		\item Support for College Algebra (co-req course, flipped, $\sim$40 student sections).
		\item Precalculus (flipped, $\sim$40 student sections).
\end{itemize}}

\section{Mentoring}
\cvitem{2020--present}{
	\textbf{Undergraduate Research}
%	\begin{itemize}
%		\item Teegan Bailey -- Covariational Reasoning
%			\begin{itemize}
%				\item 2020 UF Emerging Scholar
%			\end{itemize}
%		\item Laura Babiceanu -- Measurement Theory
%	\end{itemize}
}

%\vspace*{-4mm}

\cvitem{2019--2020}{
	\textbf{Masters of Arts in Teaching Mathematics} %- Ashley Watts
}
\cvitem{2019--2021}{
	\textbf{3\textsuperscript{rd}$\backslash$4\textsuperscript{th} year First Generation Student Life Coach}
%	\begin{itemize}
%		\item Ariel Gholar (2020--2021)
%		\item Haley Hodes (2020--2021)
%		\item Javier Moreno (2019--2020)
%	\end{itemize}
}

%\vspace*{-4mm}

\cvitem{2018--2021}{
	\textbf{University Minority Mentor Program}
%	\begin{itemize}
%		\item Anikka Bingcang (2019--2020)
%		\item Yukai He (2019--2020)
%		\item Angelica Morales (2018--2019)
%	\end{itemize}
}
%\vspace*{-4mm}

\section{Professional Leadership}
\cvitem{
	2022--present}
{\textbf{Council Member} for Mathematics Association of America Council on Teaching and Learning.}
\cvitem{
	2022--present}
	{\textbf{Subcommittee Chair} for Mathematics Association of America Subcommittee on Technologies in Mathematics Education (STME). Member since 2021.}

\cvitem{
	2022}
{\textbf{Nominating Committee Member} for the Research in Undergraduate Mathematics Education (RUME) community.}	
\cvitem{
	2020--2022}
	{\textbf{Program Committee Member} for Research in Undergraduate Mathematics Education (RUME) annual conferences.}
\cvitem{2018--2019}{
	\textbf{Huddle Leader} for the \textit{Florida College System} year-long Florida Mathematics Re-Design workgroups.
}

\section{Professional Service}
\cvitem{2022}{
	\textbf{Grant Reviewer} for the National Science Foundation.
}
\cvitem{2017--present}{
	\textbf{Journal Reviewer} for
	\begin{itemize}
		\item \textit{Educational Studies in Mathematics} since 2022;
		%\item \textit{Cogent Education} since 2021;
		\item \textit{Mathematical Thinking and Learning} since 2021;
		%\item \textit{Educational Research and Review} since 2021;
		\item \textit{International Journal of Research in Mathematics Education} since 2020;
		%\item \textit{Assessment \& Evaluation in Higher Education} since 2020;
		\item \textit{Journal of Assessment in Higher Education} since 2019;
		\item \textit{Journal of Mathematical Behavior} since 2017; and
		\item \textit{Problems, Resources, and Issues in Mathematics Undergraduate Studies} since 2017.
	\end{itemize}
} \vspace*{-3mm}

\cvitem{2017}{
	\textbf{Poster judge} for 
	\textit{Joint Mathematics Meeting, Atlanta, GA}.
}

\cvitem{2016--present}{
	\textbf{Conference Reviewer} for 
	\textit{Annual Conference on Research in Undergraduate Mathematics Education}.
}

\section{University Service}
\cvitem{2023--present}{
	\textbf{Educational Experiences Member} for the ERAU-W Quality Enhancement Plan committee. 
}
\cvitem{2023}{
	\textbf{Grant Reviewer} for ERAU Faculty Innovative Research in Science and Technology (FIRST) grant. 
}
\cvitem{2022}{
	\textbf{Grant Reviewer} for ERAU-W Faculty SEED grant. 
}
\cvitem{2022--2025}{
	\textbf{Academic Technology Committee Chair} for ERAU-W Faculty Senate. 
}
\section{College Service}
\cvitem{2023}{
	\textbf{Appeal Committee Member} for ERAU College of Arts and Sciences.
}
\cvitem{2022--present}{
	\textbf{Faculty Council Member} for ERAU-W College of Arts and Sciences.
}
\cvitem{2020--2021}{
	\textbf{Steering Committee Member} for the University of Florida College of Liberal Arts and Sciences.
}
\cvitem{2019--2021}{
	\textbf{Curriculum Committee Chair} for the University of Florida College of Liberal Arts and Sciences. \textit{Member 2019--2020.}
}
\cvitem{2018}{
	\textbf{Commencement Marshal}
	on behalf of the College of Liberal Arts and Sciences for the University of Florida's Spring 2018 and Summer 2018 undergraduate commencement ceremonies.
}

\section{Departmental Service}
\cvitem{2022--2023}{
	\textbf{Hiring Committee Member} for tenure-track candidate in Data Science for Department of Mathematics, Science, \& Technology.
}
\cvitem{2023--present}{
	\textbf{Applied Data Science Minor Coordinator} for ERAU-W Department of Mathematics, Science, \& Technology. 
}
\cvitem{2022--present}{
	\textbf{Mathematics Minor Coordinator} for ERAU-W Department of Mathematics, Science, \& Technology. 
}
\cvitem{2021--present}{
	\textbf{Course Mentor} for ERAU-W Department of Mathematics, Science, \& Technology
	\begin{itemize}
		\item MATH 111 Pre-Calculus for Aviation (2022--present)
		\item STAT 412 Probability \& Statistics (2022--present)
		\item GNED 103 Basic Mathematics (2021--2022)
		\item MATH 106 Basic Algebra \& Trigonometry (2021--2022)
	\end{itemize} 
} \vspace*{-3mm}
\cvitem{2020--2021}{
	\textbf{Hiring Committee Member} for tenure-track candidate in University of Florida College of Education. 
}
\cvitem{2017--2021}{
	\textbf{Committee Member} at University of Florida Department of Mathematics.
	\begin{itemize}
		\item Teaching Methods (\textit{Chair 2019--2021});
		\item Online Course Development; 
		\item Teaching Assistant Training; and
		\item Undergraduate Committee Lower Division.
	\end{itemize}
} \vspace*{-5mm}

%\section{Community Service}
%	\cventry{2019}
%		{Julia Robinson Mathematics Festival}
%		{}
%		{University of Florida}
%		{}
%		{Department representative for K-12 students to explore mathematics through collaborative, creative problem-solving.}

%	\cventry{2015--2017}
%		{Atlanta Science Festival}
%		{}
%		{Georgia State University}
%		{}
%		{Department representative in public celebration of local science and technology.}

\section{Professional Affiliations}
	\cvitem{2023--present}
		{\textbf{Tech in Math Ed (TIME) Organizer} for the special topic research group of SIGMAA on RUME.}
	\cvitem{2015--present}
		{\textbf{SIGMAA on RUME}: Special Interest Group of the Mathematical Association of America on Research in Undergraduate Education}
	\cvitem{2015--present}
		{\textbf{MAA}: Mathematical Association of America} 
	%\cvitem{2016--2018}
		%{\textbf{Proof Research Group} of SIGMAA on RUME}

\section{Awards and Fellowships}
	\cventry{Apr 2023}
		{Recognition Award}
		{}
		{2022-2023 ERAU-WW COAS Faculty Council Collegiality nominee.}
		{}{}
	\cventry{Apr 2023}
		{Monetary Award}
		{}
		{2022-2023 Faculty 'Superstar' Champion badge from ERAU-WW COAS Dean and Chancellor.}
		{}{}
	\cventry{2022--2023}
		{Fellowship}
		{}
		{Mathematics Association of America Project NExT. Red22 cohort.}
	{}{}
\section{Travel Grants}
	\cventry{2023}
		{External}
		{}
		{from Institute for Mathematics and its Applications University of Minnesota for Workshop on Developing Online Learning Experiments Using Doenet, May 22-26.}
		{}{}
	\cventry{2023}
		{Internal}
		{}
		{from ERAU-W Faculty Development Research Program for Conference on Research in Undergraduate Mathematics Education, February 23-25.}
		{}{}
	\cventry{2022}
		{Internal}
		{}
		{from ERAU-W Faculty Development Research Program for Joint Mathematics Meeting 2022, January 5-8}
		{}{}
	\cventry{2021}
		{Internal}
		{}
		{from UF Center for Applied Mathematics for Joint Mathematics Meeting 2021, January 6-9}
		{}{}
	\cventry{2020}
		{Internal}
		{}
		{from UF College of Liberal Arts and Sciences for Joint Mathematics Meeting 2020, January 15-18}
		{}{}
	\cventry{2017}
		{External}
		{}
		{from the American Mathematical Society for the AMS Spring 2017 Southeastern Sectional Meeting, March 10-12}
		{}{}

%	\cventry{2016}
%		{Travel Grant}
%		{}
%		{from the AMS Graduate Student Chapter at Georgia State University for the AMS Spring 2016 Southeastern Sectional Meeting, March 5-6}
%		{}{}

%	\cventry{2015}
%		{Recognition Award}
%		{}
%		{from the Georgia State University Department of Mathematics and Statistics as the 2015 Graduate Teaching Assistant of the Year}
%		{}{}

%	\cventry{2015}
%		{Recognition Award}
%		{}
%		{from Georgia State University as a 2015-2016 Who's Who Among Students in American Universities \& Colleges}
%		{}{}

%	\cventry{2015}
%		{Travel Grant}
%		{}
%		{from the American Mathematical Society Graduate Student Chapter at Georgia State University for the American Mathematical Society Spring Southeastern Regional Conference, March 27-29}
%		{}{}

\section{Notable Coursework}
	\cvitem{\textbf{Mathematics}}
		{\textbf{33 Graduate-Level Credit Hours}: Advanced Matrix Analysis I \& II, Abstract Algebra I \& II, Real Analysis I \& II,  Partial Differential Equations, Special Topics in Mathematics I \& II (Topology, Graph Theory), Directed Research (Graph Theory), Mathematical Biology. \textit{Qualifying Exams in Matrix Analysis and Abstract Algebra.} }

	\cvitem{\textbf{Mathematics Education}}
		{\textbf{15 Graduate-Level Credit Hours}: Teaching College Mathematics, Qualitative Research in Education I \& II, Epistemology of Advanced Mathematical Concepts, Learning Theories in Collegiate Mathematics Education. \textit{Qualifying Exam in Collegiate Mathematics Education.} }

	\cvitem{\textbf{Statistics}}
		{\textbf{6 Graduate-Level Credit Hours}: Mathematical Statistics, Linear Statistical Analysis.}

	\cvitem{\textbf{Data Science}}
		{\textbf{5 Coursera Courses}: Introduction to Data Science in Python, Applied Plotting, Charting \& Data Representation in Python, Applied Machine Learning in Python, Applied Text Mining in Python, Applied Social Network Analysis in Python.}

\end{document}